% !Mode:: "TeX:UTF-8"
% 请注意,此文件的编码方式一定要设置为UTF-8 无DOM模式
% 否则中文显示乱码!
% 可以用notepad++ 或ultraedit查看/更改文件编码方式
%%%%%%%%%%%%%%%%%%%%%%%%%%%%%%%%%%%%%%%%%%%%%%%5
% LaTeX常见用法示例
% 作者:http://bitjoy.net/2016/05/16/an-easy-to-use-latex-environment/

% 设置文档类型为article,纸张大小等
\documentclass[a4paper,12pt]{article}

% 插入超链接,并且去除超链接的颜色下划线等特性
\usepackage[hidelinks]{hyperref}

% 该宏包可以添加伪代码CLRS《算法导论》**第三版**模式
% MiKTeX会自动下载该宏包,
% 否则需要将clrscode3e.sty放在项目根目录下
\usepackage{clrscode3e}

% 插入数学公式
\usepackage{amsmath}

% 等号上加三角形
\usepackage{amssymb}

% 中文包
%\usepackage{xeCJK}
\usepackage{ctex} % 比xeCJK好,中文如果缩进,会严格缩进2个中文字符

% 设置中文主字体
\setCJKmainfont{SimSun}

% 设置页边距
\usepackage[top=.7in,bottom=0.8in,left=1.1in,right=1in]{geometry}

% 使用不同颜色
\usepackage{xcolor}

%%%%%%%%%%%%%为了插入代码%%%%%%%%%%%%%%%%%%%%%%%%%
\usepackage{listings}

% 预定义颜色
\definecolor{codegreen}{rgb}{0,0.6,0}
\definecolor{codegray}{rgb}{0.5,0.5,0.5}
\definecolor{codepurple}{rgb}{0.58,0,0.82}
\definecolor{backcolour}{rgb}{0.95,0.95,0.92}

% 预定于代码高亮样式
\lstdefinestyle{mystyle}{
    backgroundcolor=\color{backcolour},
    commentstyle=\itshape\color{codegreen},
    keywordstyle=\color{magenta},
    numberstyle=\tiny\color{codegray},
    stringstyle=\color{codepurple},
    basicstyle=\footnotesize,
    breakatwhitespace=false,
    breaklines=true,
    captionpos=b,
    keepspaces=true,
    numbers=left,
    numbersep=5pt,
    showspaces=false,
    showstringspaces=false,
    showtabs=false,
    tabsize=4
}

\lstdefinestyle{default}{
    commentstyle=\itshape,
    numberstyle=\tiny,
    basicstyle=\footnotesize,
    breakatwhitespace=false,
    breaklines=true,
    captionpos=b,
    keepspaces=true,
    numbers=left,
    numbersep=5pt,
    showspaces=false,
    showstringspaces=false,
    showtabs=false,
    tabsize=4
}

% 使用mystyle样式,也可以切换为default样式
\lstset{style=mystyle}
%%%%%%%%%%%%%%%%%%%%%%%%%%%%%%%%%%%%%

% 插入表格,自定义表格横竖线
% http://ctan.org/pkg/tabularx
% http://ctan.org/pkg/hhline
\usepackage{tabularx,hhline}

% 便捷实现subtable和subfigure
\usepackage{subcaption}

% for XeLaTeX symbol
\usepackage{metalogo}

% 首段自动缩进
\usepackage{indentfirst}

% 为了插入图片
\usepackage{graphicx}

% 样例图片
\usepackage{marvosym}

% 可以设置任意字体大小
\usepackage{anyfontsize}
 
% 可以固定图表位置
\usepackage{float}

% 插入的图片路径可以包含空格啦~
\usepackage[space]{grffile}

% 可以插入子图,和subcaption冲突,不推荐使用
% \usepackage{subfigure}

% 矩阵旁边标上下标
\usepackage{blkarray}% http://ctan.org/pkg/blkarray

\title{\LaTeX 示例\\ \LaTeX ~ Demo}

\author{\href{http://bitjoy.net}{bitjoy.net}\\ \href{mailto:bitjoy@qq.com}{bitjoy@qq.com}}

\begin{document}

\maketitle

% 目录页码使用罗马数字
\pagenumbering{Roman}
% 生成目录命令,需要编译两次才能正常显示
\tableofcontents

% 另起一页
\clearpage

\listoftables

% 另起一页
\clearpage

\listoffigures

% 另起一页
\clearpage

% 正文页码使用阿拉伯数字
\pagenumbering{arabic}

\section{简介}

此示例为本人常用\LaTeX 中文模板,仅供参考。请使用\XeLaTeX 编译。

段首缩进两个中文字符,请使用ctex包替换xeCJK包。段首缩进两个中文字符,请使用ctex包替换xeCJK包。段首缩进两个中文字符,请使用ctex包替换xeCJK包。段首缩进两个中文字符,请使用ctex包替换xeCJK包。段首缩进两个中文字符,请使用ctex包替换xeCJK包。段首缩进两个中文字符,请使用ctex包替换xeCJK包。段首缩进两个中文字符,请使用ctex包替换xeCJK包。段首缩进两个中文字符,请使用ctex包替换xeCJK包。

\section{数学相关}

\subsection{数学符号}

$$\alpha A  \beta B  \gamma \Gamma  \delta \Delta  \epsilon E$$
$$ \varepsilon  \zeta Z  \eta H  \theta \Theta  \vartheta$$
$$ \iota I  \kappa K  \lambda \Lambda  \mu M  \nu N$$
$$\xi \Xi  o O  \pi \Pi  \varpi  \rho P$$
$$ \varrho  \sigma \Sigma  \varsigma  \tau T  \upsilon \Upsilon$$
$$ \phi \Phi  \varphi  \chi X  \psi \Psi  \omega \Omega$$
更多详情,请点击\href{http://mohu.org/info/symbols/symbols.htm}{http://mohu.org/info/symbols/symbols.htm}

\subsection{数学公式}

Baum-Welch递归公式如下:

\begin{equation} 
\hat\mu_i^{m+1}=\frac{P(\vec Y=\vec y,X_1=i|\vec\lambda_m)}{P(\vec Y=\vec y|\vec\lambda_m)}=\gamma_1(i)
\end{equation}

\begin{equation} 
\hat a_{ij}^{m+1}=\frac{\sum\limits_{t=1}^{T-1}P(X_t=i,X_{t+1}=j|\vec Y=\vec y,\vec\lambda_m)}{\sum\limits_{t=1}^{T-1}P(X_t=i|\vec Y=\vec y,\vec\lambda_m)}\triangleq\frac{\sum\limits_{t=1}^{T-1}\xi_t(i,j)}{\sum\limits_{t=1}^{T-1}\gamma_t(i)}
\end{equation}

\begin{equation} 
\hat b_{il}^{m+1}=\frac{\sum\limits_{t=1}^TP(\vec Y=\vec y,X_t=i|\vec\lambda_m)I_{\{l\}}(y_t)}{\sum\limits_{t=1}^TP(\vec Y=\vec y,X_t=i|\vec\lambda_m)}\triangleq\frac{\sum\limits_{t=1,y_t=l}^T\gamma_t(i)}{\sum\limits_{t=1}^T\gamma_t(i)}
\end{equation}

Stable Matching Problem

\begin{equation}
\begin{array}{rrcll}
 \min & 0 &   \\
 s.t. & \sum_{i=1}^nx_{ij} & = & 1 & \text{for all $j=1,2,...,n$}  \\
      & \sum_{j=1}^nx_{ij} & = & 1 & \text{for all $i=1,2,...,n$}  \\
      & x_{ij}+x_{kl} & \leq & S_{i,j,k,l}+1 & \text{for all $i,j,k,l=1,2,...,n, i\neq k,j\neq l$}\\
      & x_{ij} & \in & \{0,1\} & \text{for all $i,j=1,2,...,n$}
\end{array}
\end{equation}

% 下面的代码和上面的代码效果相同,
% 只是下面的代码手动指定了tag{4}公式标号
% 而且最后设置\nonumber
%\[
%\begin{array}{rrcll}
% \min & 0 &   \\
% s.t. & \sum_{i=1}^nx_{ij} & = & 1 & \text{for all $j=1,2,...,n$}  \\ \tag{4}
%      & \sum_{j=1}^nx_{ij} & = & 1 & \text{for all $i=1,2,...,n$}  \\
%      & x_{ij}+x_{kl} & \leq & S_{i,j,k,l}+1 & \text{for all $i,j,k,l=1,2,...,n, i\neq k,j\neq l$}\\
%      & x_{ij} & \in & \{0,1\} & \text{for all $i,j=1,2,...,n$}
%\end{array} \nonumber
%\]

Subsequence Counting

\begin{equation}
dp[i][j]=
\begin{cases}
dp[i-1][j] & \text{if $S[i]\neq T[j]$}\\
dp[i-1][j]+dp[i-1][j-1] & \text{if $S[i]=T[j]$}\\
\end{cases}
\end{equation}

如果不需要公式标号,可以把equation环境去掉,换成\$\$ \$\$或$\backslash[~\backslash]$。

Linear Program

\begin{equation}
\begin{array}{rrrrrrrrl}
 \max & 3x_1 + x_2 + 2x_3&   \\
 s.t. & x_1 + x_2 + 3x_3 \leq 30 &  \\
      & 2x_1 + 2x_2 + 5x_3 \leq 24 & \\
      & 4x_1 + x_2 + 2x_3 \leq 36 & \\
      & x_1, x_2, x_3 \geq 0
\end{array}
\end{equation}

We have:

\begin{gather}
A=\begin{bmatrix}
1 & 1 & 3\\
2 & 2 & 5\\
4 & 1 & 2
\end{bmatrix}~~~~
b=\begin{bmatrix}
30\\
24\\
36
\end{bmatrix}~~~~
c=\begin{bmatrix}
3\\
1\\
2
\end{bmatrix}
\end{gather}

After running my implementation, we get:
\begin{gather}
x=\begin{bmatrix}
8\\
4\\
0
\end{bmatrix}
\end{gather}

如果不需要矩阵标号,可以把gather改为gather*。


Jacobi 矩阵:
\[
\begin{blockarray}{r@{}cccccccc}
\begin{block}{r(ccccccc)c}
                 & 1 & \cdots & 0 & \cdots & 0 & \cdots & 0 \\
                 & \vdots & \ddots & \vdots & & \vdots & & \vdots \\
                 & 0 & \cdots & c & \cdots & -s & \cdots & 0 & i \\
J(i,j,\theta)={} & \vdots & & \vdots & \ddots & \vdots & & \vdots \\
                 & 0 & \cdots & s & \cdots & c & \cdots & 0 & j \\
                 & \vdots & & \vdots & & \vdots & \ddots & \vdots \\
                 & 0 & \cdots & 0 & \cdots & 0 & \cdots & 1 \\
\end{block}
                 & & & i & & j & \\
\end{blockarray}
\]


\section{计算机相关}

\subsection{伪代码}

\textit{Introduction to Algorithms, third edition} page 631, growing a minimum spanning tree.

\begin{codebox}
\Procname{MST-KRUSKAL($G,w$)}
\li $A=\varnothing$
\li \For each vertex $v\in G.V$
	\Do
\li MAKE-SET($v$)
	\End
\li sort the edges of $G.E$ into nondecreasing order by weight $w$
\li \For each edge $(u,v)\in G.E$, taken in nondecreasing order by weight
	\Do
\li \If FIND-SET($u$)$\neq$FIND-SET($v$)
	\Then
\li $A=A\cup\{(u,v)\}$
\li UNION($u,v$)
	\End
	\End
\li \Return A
\end{codebox}


\subsection{代码高亮}

\begin{lstlisting}[language=c++]
#include<iostream>
using namespace std;
int main(){
	cout<<"Hello World!"<<endl;
	return 0;
}
\end{lstlisting}


\section{图表相关}

\subsection{常规图表}

\begin{figure}[H] % H: fix the figure position
\centering
%\includegraphics[width=0.3\textwidth]{moustache-male-face-emoticon-symbol.eps}
\fontsize{150}{60}\selectfont\Ladiesroom
\caption{Lady symbol}\label{fig:1}
\end{figure}

\begin{table}[!ht]
\centering
\begin{tabular}{c|c|c|c|c} 
$Element$ & $S_1$ & $S_2$ & $S_3$ & $S_4$ \\
\hline
\hline
0 & 0 & 1 & 0 & 1 \\
1 & 0 & 1 & 0 & 0 \\
2 & 1 & 0 & 0 & 1 \\
3 & 0 & 0 & 1 & 0 \\
4 & 0 & 0 & 1 & 1 \\
5 & 1 & 0 & 0 & 0 \\
\end{tabular}
\caption{Matrix for Exercise 3.3.3(\textit{Mining of Massive Datasets})}
\end{table}


\subsection{特殊图表(包含子图表)}

\begin{table}[htb]
	\begin{subtable}{.5\linewidth}\centering
    \begin{tabular}{|c|c|c|c|c|}
     \hline
     & TA & TB\\
     \hline
     TA & & \\
     \hline
     TB & & \\
     \hline\hline
     初概率 & & \\
     \hline
    \end{tabular}
	\caption{初始概率$\vec \mu$和转移概率矩阵$A$}\label{tab:1a}
	\end{subtable}
	\begin{subtable}{.5\linewidth}\centering
    \begin{tabular}{|c|c|c|c|c|} 
      \hline
       & A & C & T & G \\
      \hline
       TA & & & & \\
      \hline
       TB & & & & \\
      \hline
    \end{tabular}
	\caption{发射概率矩阵$B$}\label{tab:1b}
	\end{subtable}
\caption{需要求解的模型参数$\vec\lambda$}\label{tab:1}
\end{table}


\begin{figure}[H]
	\begin{subfigure}[b]{.5\linewidth}
	\centering
	\fontsize{150}{60}\selectfont\Ladiesroom
	\caption{Lady 1}\label{fig:21a}
	\end{subfigure} % 注意不能有空行,否则会换行
	\begin{subfigure}[b]{.5\linewidth}
	\centering
	\fontsize{150}{60}\selectfont\Ladiesroom
	\caption{Lady 2}\label{fig:2b}
	\end{subfigure}
	
	\begin{subfigure}[b]{.5\linewidth}
	\centering
	\fontsize{150}{60}\selectfont\Ladiesroom
	\caption{Lady 3}\label{fig:2c}
	\end{subfigure} % 注意不能有空行,否则会换行
	\begin{subfigure}[b]{.5\linewidth}
	\centering
	\fontsize{150}{60}\selectfont\Ladiesroom
	\caption{Lady 4}\label{fig:2d}
	\end{subfigure}
\caption{Four ladies}\label{fig:2}
\end{figure}

使用float宏包,然后用[H]标签可以固定图表的位置。

\begin{lstlisting}[language=c++]
\begin{figure}[H]
\end{figure}
\end{lstlisting}

\end{document}
