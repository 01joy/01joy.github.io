% !Mode:: "TeX:UTF-8"
% 请注意,此文件的编码方式一定要设置为UTF-8 无DOM模式
% 否则中文显示乱码!
% 可以用notepad++ 或ultraedit查看/更改文件编码方式

\documentclass[a4paper,12pt]{article}

%插入超链接,并且去除超链接的颜色下划线等特性
\usepackage[hidelinks]{hyperref}

%该宏包可以添加伪代码CLRS《算法导论》**第三版**模式
%clrscode3e.sty放在项目根目录下即可
\usepackage{clrscode3e}

%插入数学公式
\usepackage{amsmath}

%中文包
\usepackage{xeCJK}
\setCJKmainfont{SimSun}

%设置页边距
\usepackage[top=1in,bottom=0.8in,left=1.1in,right=1in]{geometry}

\usepackage{xcolor}

%为了插入代码
\usepackage{listings}

\definecolor{codegreen}{rgb}{0,0.6,0}
\definecolor{codegray}{rgb}{0.5,0.5,0.5}
\definecolor{codepurple}{rgb}{0.58,0,0.82}
\definecolor{backcolour}{rgb}{0.95,0.95,0.92}

\lstdefinestyle{mystyle}{
    backgroundcolor=\color{backcolour},
    commentstyle=\itshape\color{codegreen},
    keywordstyle=\color{magenta},
    numberstyle=\tiny\color{codegray},
    stringstyle=\color{codepurple},
    basicstyle=\footnotesize,
    breakatwhitespace=false,
    breaklines=true,
    captionpos=b,
    keepspaces=true,
    numbers=left,
    numbersep=5pt,
    showspaces=false,
    showstringspaces=false,
    showtabs=false,
    tabsize=4
}

\lstdefinestyle{default}{
    backgroundcolor=\color{backcolour},
    commentstyle=\itshape,
    numberstyle=\tiny,
    basicstyle=\footnotesize,
    breakatwhitespace=false,
    breaklines=true,
    captionpos=b,
    keepspaces=true,
    numbers=left,
    numbersep=5pt,
    showspaces=false,
    showstringspaces=false,
    showtabs=false,
    tabsize=4
}

\lstset{style=mystyle}

%为了插入图片
%\usepackage{graphicx}

%首段缩进
\usepackage{indentfirst}

\title{Assignment 4\\Algorithm Design and Analysis}

\author{\href{http://bitjoy.net}{bitjoy.net}}
\date{November 27, 2015}

\begin{document}

\maketitle

I choose problem 1,2,4,7.

\section*{1\quad Linear-inequality feasibility}
\subsection*{\textnormal{1.1\quad Linear programming => linear-inequality feasibility problem}}

Suppose we have an algorithm for linear programming:

\[
\begin{array}{rrrrrrrrl}
 \min & \mathbf{c^T x }&   \\
 s.t. & \mathbf{A x \leq  b} &  \\ \tag{1}
      & \mathbf{x \geq 0} & \\
\end{array} \nonumber
\]

We just change the objective in (1) like this:

\[
\begin{array}{rrrrrrrrl}
 \min & \mathbf{0}&   \\
 s.t. & \mathbf{A x \leq  b} &  \\ \tag{2}
      & \mathbf{x \geq 0} & \\
\end{array} \nonumber
\]

and run linear programming again. If (2) has optimal solution, then linear-inequality

\[
\begin{array}{rrrrrrrrl}
& \mathbf{A x \leq  b} &  \\ \tag{3}
& \mathbf{x \geq 0} & \\
\end{array} \nonumber
\]

is feasible, otherwise infeasible. (1) only costs polynomial time.

\subsection*{\textnormal{1.2\quad Linear-inequality feasibility => linear programming}}

Suppose we can get the feasible solution of primal problem and its dual problem, say a and b, but neither is optimal(c). As we know, the optimal solution of primal problem must be between these two values, say a>c>b. Then we get the median(m) of a and b, and add an inequality obj>m. If new inequality is feasible, then optimal should be [m,a], otherwise [b,m]. We check feasibility of new inequality recursively until the interval is small enough.

\section*{2\quad Airplane Landing Problem}
Let $x_1,x_2,...,x_n$ be the exact landing time of each airplane, the LP formulation of this problem should be:
\[
\begin{array}{rcl}
\max & \min_{i=2...n} (x_i-x_{i-1}) &   \\ \tag{4}
s.t. & s_i \leq x_i \leq t_i & $for $i$ in 1...n$\\
\end{array} \nonumber
\]

According to Robert Fourer's book \emph{Optimization Models}\footnote{\url{http://www.4er.org/CourseNotes/Book\%20A/A-III.pdf}}, (4) is equivalent to

\[
\begin{array}{rcl}
\max & z\\
s.t. & z \leq x_i-x_{i-1} &   $for $i$ in 2...n$\\ \tag{5}
     & s_i \leq x_i \leq t_i & $for $i$ in 1...n$\\
\end{array} \nonumber
\]

For the example in problem description, if the time window of landing three airplanes are [1,60], [80,100] and [120,140], use GLPK to solve it:

\begin{lstlisting}
var x1 >= 1, <=60;
var x2 >= 80, <= 100;
var x3 >=120, <=140;
var z;
maximize gap: z;
s.t. gap1: z<=x2-x1;
s.t. gap2: z<=x3-x2;
end;
\end{lstlisting}

The optimal result is $z=60$ and $x_1=1, x_2=80, x_3=140$. Thus, they land at 10:00, 11:20, 12:20 respectively.

\section*{4\quad Gas Station Placement}

Let $x_1,x_2,...,x_n$ be the place of each gas station, as $d_1,d_2,...,d_n$ and $r$ have been given, we can get the LP formulation like this:

\[
\begin{array}{rcl}
\min & \max_{i=2...n} (x_i-x_{i-1}) &   \\ \tag{6}
s.t. & |x_i-d_i|\leq r & $for $i$ in 1...n$\\
\end{array} \nonumber
\]

Similarly, (6) is equivalent to

\[
\begin{array}{rcl}
\min & z\\
s.t. & z\geq x_i-x_{i-1} & $for $i$ in 2...n$  \\ \tag{7}
     & |x_i-d_i|\leq r & $for $i$ in 1...n$\\
\end{array} \nonumber
\]

\section*{7\quad Simplex Algorithm}

Consider the following linear program in standard form:

\[
\begin{array}{rrrrrrrrl}
 \max & \mathbf{c^T x }&   \\
 s.t. & \mathbf{A x \leq  b} &  \\ \tag{8}
      & \mathbf{x \geq 0} & \\
\end{array} \nonumber
\]

I have implemented the Simplex Algorithm in Python 3 according to Chapter 29 of \emph{Introduction to Algorithms}:

\begin{lstlisting}[language=python]
# -*- coding: utf-8 -*-
"""
Created on Thu Nov 26 18:44:28 2015

@author: czl
"""

import numpy as np

class SIMPLEX:
    m = 0
    n = 0

    def __init__(self):
        pass

    def INITIALIZE(self, A, b, c):
        k = b.argmin()
        if b[k] >= 0: # Note, I only implemented the easy case.
            AA = np.zeros((self.m + self.n + 1, self.m + self.n + 1))
            bb = np.array([0.0] * (self.m + self.n + 1)) # 0.0 for float64
            cc = np.array([0.0] * (self.m + self.n + 1)) # 0.0 for float64
            AA[self.n + 1 : self.n + self.m + 1, 1 : self.n + 1] = A
            bb[self.n + 1 : self.n + self.m + 1] = b
            cc[1 : self.n + 1] = c
            return(np.arange(1, self.n + 1, 1), np.arange(self.n + 1, self.n + self.m + 1, 1), AA, bb, cc, 0)

    def PIVOT(self, N, B, A, b, c, v, l, e):
        AA = np.zeros((self.m + self.n + 1, self.m + self.n + 1))
        b[e] = b[l] / A[l][e]
        for j in N:
            if j != e:
                AA[e][j] = A[l][j] / A[l][e]
        AA[e][l] = 1 / A[l][e]
        for i in B:
            if i != l:
                b[i] = b[i] - A[i][e] * b[e]
                for j in N:
                    if j != e:
                        AA[i][j] = A[i][j] - A[i][e] * AA[e][j]
                AA[i][l] = - A[i][e] * AA[e][l]
        v = v + c[e] * b[e]
        for j in N:
            if j != e:
                c[j] = c[j] - c[e] * AA[e][j]
        c[l] = - c[e] * AA[e][l]
        c[e] = 0 # clear c of enter
        b[l] = 0 # clear b of leave
        NN = np.delete(N, np.where( N == e)[0][0])
        NN = np.append(NN,l)
        BB = np.delete(B, np.where( B == l)[0][0])
        BB = np.append(BB,e)
        return(NN, BB, AA, b, c, v)


    def SOLVE(self, A, b, c):
        self.m, self.n = A.shape
        N, B, A, b, c, v = self.INITIALIZE(A, b, c)
        while c.max() > 0:
            d = np.array([float('inf')] * (self.m + self.n + 1))
            e = c.argmax() # choose index of max c
            for i in B:
                if A[i][e] > 0:
                    d[i] = b[i] / A[i][e]
            l = d.argmin()
            if d[l] == float('inf'):
                return -2 # unbounded
            else:
                N, B, A, b, c, v = self.PIVOT(N, B, A, b, c, v, l, e)
        x = np.array([0] * self.n)
        for i in range(self.n):
            if i + 1 in B:
                x[i] = b[i + 1]
        return x


if __name__ == "__main__":
    A = np.array([[1,1,3],
                  [2,2,5],
                  [4,1,2]])
    b = np.array([30,24,36])
    c = np.array([3,1,2])
    s = SIMPLEX()
    x = s.SOLVE(A, b, c)

\end{lstlisting}

As I was fully occupied with lots of things, I only implemented the easy case in finding an initial solution, see function \textbf{INITIALIZE}. But I will finish another case in the future.

Here is a test example:

\[
\begin{array}{rrrrrrrrl}
 \max & 3x_1 + x_2 + 2x_3&   \\
 s.t. & x_1 + x_2 + 3x_3 \leq 30 &  \\ \tag{9}
      & 2x_1 + 2x_2 + 5x_3 \leq 24 & \\
      & 4x_1 + x_2 + 2x_3 \leq 36 & \\
      & x_1, x_2, x_3 \geq 0
\end{array} \nonumber
\]

We have:

\begin{gather*}
A=\begin{bmatrix}
1 & 1 & 3\\
2 & 2 & 5\\
4 & 1 & 2
\end{bmatrix}~~~~
b=\begin{bmatrix}
30\\
24\\
36
\end{bmatrix}~~~~
c=\begin{bmatrix}
3\\
1\\
2
\end{bmatrix}
\end{gather*}

After running my implementation, we get:
\begin{gather*}
x=\begin{bmatrix}
8\\
4\\
0
\end{bmatrix}
\end{gather*}
The optimal solution is $z=3x_1+x_2+2x_3=28$, the result is the same as GLPK's.

\end{document}
