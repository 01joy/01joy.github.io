% !Mode:: "TeX:UTF-8"
% 请注意,此文件的编码方式一定要设置为UTF-8 无DOM模式
% 否则中文显示乱码!
% 可以用notepad++ 或ultraedit查看/更改文件编码方式

\documentclass[a4paper,12pt]{article}

%插入超链接,并且去除超链接的颜色下划线等特性
\usepackage[hidelinks]{hyperref}

%该宏包可以添加伪代码CLRS《算法导论》**第三版**模式
%clrscode3e.sty放在项目根目录下即可
\usepackage{clrscode3e}

%插入数学公式
\usepackage{amsmath}

%中文包
\usepackage{xeCJK}
\setCJKmainfont{SimSun}

%设置页边距
\usepackage[top=1in,bottom=0.8in,left=1.1in,right=1in]{geometry}

\usepackage{xcolor}

%为了插入代码
\usepackage{listings}

\definecolor{codegreen}{rgb}{0,0.6,0}
\definecolor{codegray}{rgb}{0.5,0.5,0.5}
\definecolor{codepurple}{rgb}{0.58,0,0.82}
\definecolor{backcolour}{rgb}{0.95,0.95,0.92}

\lstdefinestyle{mystyle}{
    backgroundcolor=\color{backcolour},
    commentstyle=\itshape\color{codegreen},
    keywordstyle=\color{magenta},
    numberstyle=\tiny\color{codegray},
    stringstyle=\color{codepurple},
    basicstyle=\footnotesize,
    breakatwhitespace=false,
    breaklines=true,
    captionpos=b,
    keepspaces=true,
    numbers=left,
    numbersep=5pt,
    showspaces=false,
    showstringspaces=false,
    showtabs=false,
    tabsize=4
}

\lstdefinestyle{default}{
    backgroundcolor=\color{backcolour},
    commentstyle=\itshape,
    numberstyle=\tiny,
    basicstyle=\footnotesize,
    breakatwhitespace=false,
    breaklines=true,
    captionpos=b,
    keepspaces=true,
    numbers=left,
    numbersep=5pt,
    showspaces=false,
    showstringspaces=false,
    showtabs=false,
    tabsize=4
}

\lstset{style=mystyle}

%为了插入图片
%\usepackage{graphicx}

%首段缩进
\usepackage{indentfirst}

\usepackage{array}

\title{Assignment 4 Supplement\\Algorithm Design and Analysis}

\author{\href{http://bitjoy.net}{bitjoy.net}}
\date{January 20, 2016}

\begin{document}

\maketitle

\section*{3\quad Interval Scheduling Problem}
假设有$n$门课,$m$个教室。用$x_{ij}$表示第$i$门课是否安排在第$j$个教室,$x_{ij}=1$ 为Yes,$x_{ij}=0$为No。 则有以下两个限制条件:
\begin{enumerate}
  \item 一门课最多安排在一个教室里
  \item 任何时刻,一个教室只能有一门课在进行
\end{enumerate}
首先对所有课程的结束时间排序,使得$F_1\leq F_2\leq ...\leq F_n$,然后求解如下LP公式:

%http://tex.stackexchange.com/questions/12703/how-to-create-fixed-width-table-columns-with-text-raggedright-centered-raggedlef
\newcolumntype{L}[1]{>{\raggedright\let\newline\\\arraybackslash\hspace{0pt}}m{#1}}
\newcolumntype{C}[1]{>{\centering\let\newline\\\arraybackslash\hspace{0pt}}m{#1}}
\newcolumntype{R}[1]{>{\raggedleft\let\newline\\\arraybackslash\hspace{0pt}}m{#1}}

\begin{center}
\begin{tabular}{L{2.5cm}rcllR{1.8cm}}
& max & $\sum\limits_{i=1}^n\sum\limits_{j=1}^mx_{ij}$ & & & \\
& $\sum\limits_{j=1}^mx_{ij}$ & $\leq$ & 1 & for all $i$ & (1) \\
& $F_i(x_{ik}+x_{jk}-1)$ & $\leq$ & $S_j$ & for all $1\leq i<j\leq n,k$ & (2)\\
& $x_{ij}$ & $\in$ & {0,1} & for all $i,j$  & (3)\\
\end{tabular}
\end{center}

第(1)个式子很好理解,把一门课在所有教室的安排情况加起来,不大于1表示最多只能在一个教室。

第(2)个式子对应第2个条件,如果第$i$和第$j$门课都安排在第$k$个教室,则要满足这两门课的上课时间没有重叠,即当$x_{ik}=x_{jk}=1$时,有$F_i\leq S_j$;如果$x_{ik}$和$x_{jk}$有一个为0或全为0,则显然满足条件。

如果$F_i$和$S_i$也为整数,则得到一个Totally Unimodular Matrix\footnote{\url{http://www.imada.sdu.dk/~marco/Teaching/Fall2009/DM204/Slides/TUM-lau.pdf}},即使将ILP松弛成LP,得到的解也是整数。(助教:I don't konw!)

\section*{5\quad Stable Matching Problem}
设$x_{ij}$表示男$i$和女$j$的配对情况,1表示稳定配对,0表示不配对。

\begin{enumerate}
  \item ILP公式如下:
\[
\begin{array}{rrcll}
 \min & 0 &   \\
 s.t. & \sum_{i=1}^nx_{ij} & = & 1 & \text{for all $j=1,2,...,n$}  \\ \tag{4}
      & \sum_{j=1}^nx_{ij} & = & 1 & \text{for all $i=1,2,...,n$}  \\
      & x_{ij}+x_{kl} & \leq & S_{i,j,k,l}+1 & \text{for all $i,j,k,l=1,2,...,n, i\neq k,j\neq l$}\\
      & x_{ij} & \in & \{0,1\} & \text{for all $i,j=1,2,...,n$}
\end{array} \nonumber
\]
前两个限制条件表示一个女生只能和一个男生配对以及一个男生只能和一个女生配对。

第三个限制条件表示,如果$S_{i,j,k,l}=1$,则$<i,j>$和$<k,l>$一定是稳定的,即$x_{ij}=1,x_{kl}=1$;如果$S_{i,j,k,l}=0$,则$<i,j>$和$<k,l>$至少有一个不稳定。
  \item ILP公式如下:
\[
\begin{array}{rrcll}
 \min & 0 &   \\
 s.t. & \sum_{i=1}^nx_{ij} & = & 1 & \text{for all $j=1,2,...,n$}  \\ \tag{5}
      & \sum_{j=1}^nx_{ij} & = & 1 & \text{for all $i=1,2,...,n$}  \\
      & x_{ik}+x_{lj} & \leq & 3-p_{l,k,j}-q_{k,l,i} & \text{for all $i,j,k,l=1,2,...,n, k\neq j,l\neq i$}\\
      & x_{ij} & \in & \{0,1\} & \text{for all $i,j=1,2,...,n$}
\end{array} \nonumber
\]
前两个限制条件和1问相同。第三个限制条件表示:在$i,j,k,l$这四个人中,如果$p_{l,k,j}=q_{k,l,i}=1$,则$l$更喜欢$k$,同时$k$也更喜欢$l$,所以$<l,k>$ 肯定配对成功,$i$ 和$j$对其没影响,即$x_{ik}=x_{lj}=0$。

第三个限制条件也可改为$x_{ij}+x_{kl}\leq 2-p_{i,l,j}q_{j,k,i}$。
\end{enumerate}

\section*{6\quad Duality}

For simplicity, we can assume that $(u,v)$ denotes the arc $u\rightarrow v$. Then the primal can be rewritten and corrected as:

\[
\begin{array}{rrcll}
 \max/\min & 0 &   \\
 s.t. & \sum_{i=1}^kf_i(u,v) & \leq & c(u,v) & \text{for each $(u,v)$}  \\ \tag{6}
      & \sum_{v,(u,v)\in E}f_i(u,v)-\sum_{v,(v,u)\in E}f_i(v,u) & = & 0 & \text{for each $i$ and $u\in V \setminus \{s_i,t_i\}$}  \\
      & \sum_{v,(s_i,v)\in E}f_i(s_i,v)-\sum_{v,(v,s_i)\in E}f_i(v,s_i) & = & d_i & \text{for each $i$}\\
      & f_i(u,v) & \geq & 0 & \text{for each $i,(u,v)$}
\end{array} \nonumber
\]

If we use $x_{uv}$ to denote the first constraints, $y_{iu}$ the second and third constraints, then the duality is:
\[
\begin{array}{rrcll}
 \min & c(u,v)x_{uv}+d_iy_{is_i} &   \\
 s.t. & x_{uv}+y_{iu}-y_{iv} & \geq & 0 & \text{for all $i$ and $u\neq t_i,v\neq t_i$}  \\ \tag{7}
      & x_{ut_i}+y_{iu} & \geq & 0 & \text{for all $i,(u,t_i)$}  \\
      & x_{t_iv}-y_{iv} & \geq & 0 & \text{for all $i,(t_i,v)$}\\
      & x_{uv} & \geq & 0 & \text{for all $(u,v)$}
\end{array} \nonumber
\]
or
\[
\begin{array}{rrcll}
 \max & c(u,v)x_{uv}+d_iy_{is_i} &   \\
 s.t. & x_{uv}+y_{iu}-y_{iv} & \leq & 0 & \text{for all $i$ and $u\neq t_i,v\neq t_i$}  \\ \tag{8}
      & x_{ut_i}+y_{iu} & \leq & 0 & \text{for all $i,(u,t_i)$}  \\
      & x_{t_iv}-y_{iv} & \leq & 0 & \text{for all $i,(t_i,v)$}\\
      & x_{uv} & \leq & 0 & \text{for all $(u,v)$}
\end{array} \nonumber
\]
\section*{8\quad LP And Iterative Algorithm}
Too cumbersome to finish it.
\end{document}
